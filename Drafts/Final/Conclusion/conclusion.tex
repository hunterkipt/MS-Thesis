\chapter{Conclusion}

In this thesis, we have proposed new methods for extracting fingerprints used in detecting frame deletion in modern video as well as an expanded classification framework for more robust classification on real world data. To do this, we first examined previous work on detecting frame deletion in MPEG-2.  We performed an experiment to show that the methodology used for detecting frame deletion in MPEG-2 resulted in detection rates below those obtained in the original research for MPEG-2, and found that due to the differences in motion compensation and estimation found in H.264, an expanded model for the fingerprint signal was needed. We then used this model to alter the prediction error extraction technique to infer the source frame of all macroblocks for every P-frame in the video. Using this inferred prediction error, we were able to isolate the fingerprint signal in an H.264 video. Additionally, we proposed using autoregressive models and additional features to capture statistical variations between videos, and thus shape the decision surface of a Support Vector Machine classifier to be independent of the amount of motion in a video or its scene content.

Through a series of experiments we have evaluated the performance of our proposed prediction error extraction methodology and proposed feature set independently. Our results show that both our proposed prediction error extraction methodology and feature set yield increases to the probability of detecting frame deletion in H.264, particularly at low false alarm rates. This was verified on a diverse set of video captured from multiple camera models and devices. Using both of these techniques together yield an almost 80\% probability of detection at a 10\% false alarm rate, while the methodology proposed for use on MPEG-2 video was unable to accurately identify the set of videos with frame deletion. We also considered the effects of introducing recompressed videos to our proposed detector. We showed that while it is difficult to distinguish between videos with frame deletion and recompressed videos, simple data processing can allow our proposed detection methods to begin to identify their differences. As such, additional work can be done to enable this distinction, potentially with new features or the use of more advanced classification techniques.