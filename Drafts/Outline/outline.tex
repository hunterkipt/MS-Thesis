
\documentclass[12pt]{article}%
\usepackage{graphics, graphicx, cite, fancybox, setspace}
\usepackage{color}
\usepackage{boxedminipage}
\usepackage{amsfonts,amssymb,amsmath}
\usepackage{url}
\usepackage{wrapfig,sectsty}
\usepackage[letterpaper, left=1in, right=1in, top=1in, bottom=1in]{geometry}
\usepackage{multirow}
\usepackage{times}
\usepackage{verbatim}
\usepackage{epsfig}
\usepackage{subcaption}

%
\usepackage{array}
\usepackage{rotating}


\usepackage{enumitem}
\usepackage{float}
\usepackage{titlesec}







%%%%%%%%%%%%%%%%%%%%%%%%%%%%%%%%%%%%%%%%%%%%%%%%%%%%%%%


\titleformat{\section}
  {\normalfont\fontsize{14}{1em}\bfseries}{\thesection}{1em}{}

%%%%%%%%%%%%%%%%%%%%%%%%%%%%%%%%%%%%%%%%%%%%%%%%%%%%%%%

\def\bi{\begin{itemize}     % Begin Itemize
\vspace{-0.5em}\setlength\itemsep{0em}}

\def\be{\begin{enumerate}    % Begin Enumerate
\vspace{-0.5em}\setlength\itemsep{0em}}

%%%%%%%%%%%%%%%%%%%%%%%%%%%%%%%%%%%%%%%%%%%%%%%%%%%%%%%


\begin{document}

\begin{center}
{\LARGE Master's Thesis Outline}\\
\vspace{0.5em}
{\Large Hunter Kippen}
\vspace{0.5em}
\end{center}


\be
  \item Introduction
  \be
    \item Motivate the need for digital media forensics.
    \be
      \item Explain why determining authenticity is important.
      \item Explain how easy it is to make forgeries (especially of the frame deletion kind).
      \item Show that tools need to be developed to tackle these problems, since we cannot tell by inspection.
    \end{enumerate}
    
    \item Sample various topics in digital media forensics.
    \be
      \item Source Identification
      \item Forgery Detection
    \end{enumerate}
    
    \item Drill down into the problem of video tampering.
    \be
      \item Explain that removing frames from video can change context (news media).
      \item Explain that there is a previous body of research that attempted to solve this problem.
      \be
        \item The research was done on MPEG-2 video.
        \item Since the research was done, most videos recorded use more advanced encoding.
        \item The methods used previously may not work anymore, and if they do, they are now suboptimal.
      \end{enumerate}
    \end{enumerate}
    
    \item Explain my proposed solution and motivate the topic of the thesis.
  \end{enumerate}
  
  \item Related Work/Background
  \be
    \item Subsection: Video Encoding
    \be      
      \item Talk about video encoding basics. Particularly that of MPEG-2.
    \end{enumerate}
    \item Subsection: Prior Frame Deletion Detection Research
    \be
      \item Talk about work done on MPEG-2.
      \be
        \item Wang and Farid's paper.
        \item Dr. Stamm's paper.
        \item Other work done. Find papers that cite Farid's original.
      \end{enumerate}
    \end{enumerate}
    \item Mauro Barni's Paper
    \item Brief mention of why prior research doesn't solve the problem.
    \item Lightly Motivate Problem
  \end{enumerate}
  
  \item Problem Formulation
  \be
    \item Why do we need a new frame deletion detection algorithm?
    \be
      \item Limitations in assumptions made in Farid's and Stamm's papers
      \item Link definition of fingerprint signal to why it might not work in H.264.
      \item It is not obvious that these previous methods will work in H.264.
    \end{enumerate}
    \item Subsection: Frame/segment deletion traces.
    \item Subsection: Frame/segment deletion detection.
    \be
      \item Set up as a classification problem.
    \end{enumerate}
    \item Elaborate assumptions of double compression.
    \item Assumptions of deletion from beginning of video.
    \item Assumptions of number of frames deleted (may not be necessary depending on dataset used).
    \item Explain mathematical definition of fingerprint signal.
  \end{enumerate}  
  
  \item Proposed Approach
  \be
    \item Subsection: Prediction Error Sequence Extraction.
    \be
      \item Say how prediction error sequence was extracted in prior work
      \item show experiment demonstrating that frame deletion detection fails when this old sequence is used with H.264
      \item Describe how \emph{you} propose extracting prediction error sequence.
    \end{enumerate}
    \item Subsection: Proposed Detection Algorithm
    \be
      \item Why you do it.
      \item What you do.
      \item Rationale for additional detection features.
      \item Rationale for AR modeling.
    \end{enumerate}
  \end{enumerate}
  
  \item Experimental Results
  \be
    \item One camera model.
    \be
      \item Basic Accuracy for all methods.
      \item Accuracy if Reencode GOP doesn't match original.
      \item Accuracy for varying numbers of frames deleted.
      \item Accuracy for uniformly selected numbers of frames deleted.
    \end{enumerate}
    
    \item Multiple Camera Models.
    \be
      \item Basic Accuracy for all methods (Reencode GOP matches original GOP for each camera).
      \item Accuracy as number of camera models in training set are increased, testing on an unknown camera.
      \item Accuracy as number of frames deleted from each camera's set is different.
      \item Accuracy vs number of frames in a sequence in a video.
    \end{enumerate}
  \end{enumerate}
  
  \item Conclusion


\end{enumerate}


% Bib/refs go here if used

\begin{comment}
\pagebreak
%\bibliographystyle{plain}
\bibliographystyle{myIEEE}
\bibliography{StammRefs}
%\bibliography{StammRefs,kandasamy_v2,kandasamy}
\end{comment}


\end{document} 